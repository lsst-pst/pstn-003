

\documentclass[PST,authoryear,toc]{lsstdoc}
% lsstdoc documentation: https://lsst-texmf.lsst.io/lsstdoc.html

% Package imports go here.

% Local commands go here.

\graphicspath{ {./images/} {./figures/} {../images} }
% To add a short-form title:
% \title[Short title]{Title}
\title{Discussion of Object vs. Source table queries and data distribution}

% Optional subtitle
% \setDocSubtitle{A subtitle}

\author{%
William O'Mullane, Fritz Mueller
}

\setDocRef{PSTN-003}

\date{\today}

% Optional: name of the document's curator
% \setDocCurator{The Curator of this Document}

\setDocAbstract{%
AMCL asked about distribution of various catalog products. They also suggested
potential usage levels for these products. This brings questions such as:
Qserv is built to allow SQL queries on sources, but will most queries be on object?
This note discuses data access, interaction, and distribution.
}

% Change history defined here.
% Order: oldest first.
% Fields: VERSION, DATE, DESCRIPTION, OWNER NAME.
% See LPM-51 for version number policy.
\setDocChangeRecord{%
  \addtohist{1}{YYYY-MM-DD}{Unreleased.}{William O'Mullane, Fritz Mueller}
}

\begin{document}

% Create the title page.
% Table of contents is added automatically with the "toc" class option.

\mkshorttitle
%switch to \maketitle if you wan the title page and toc


% ADD CONTENT HERE ... a file per section can be good for editing
\section{Introduction} \label{sec:intro}
 Writing concise and testable requirements is very difficult.
 Writing requirements in 2005 for a system to run in 2022 is extremely difficult but is the case for DM.
Assumptions are made about requirements and how to implement them, but
the perspective of the requirement writer and implementer are usually not identical.
Over a long period this could diverge significantly and  choices made years ago may not be so valid anymore.
So we should continually challenge requirements and ensure they are still valid and that we are interpreting them correctly.


\section{The catalog access question}\label{sec:cat}
The DataBase requirements are in \citeds{LSE-61} and \citeds{LDM-555}.
A series of use cases has been collected in \citeds{DMTN-086}.
It is interesting to consider the size of tables summarized in \tabref{tab:sizes}.


\input{images/sizetab.tex} % this is in the images git repo

It is also worth considering the usage for those tables as in \tabref{tab:use}. {\bf PST should closely examine this table, which came from the AMCL and consider if the numbers are correct/plausible.}
\input{images/usetab.tex} % this is in the images git repo

\subsection{Baseline approach to catalog interaction SQL/ Qserv}
\label{sec:qserv}
Qserv is a custom massively parallel database built by LSST(SLAC) for LSST.
This has been built on the assumption (requirement) that astronomy on the catalog will be done as queries.
Qserv provides SQL access (with some query limitations) to all catalogs, including visits e.g. force photometry/light curves as depicted in \figref{fig:catopt1}.
Some implementation remains for Qserv, e.g. MyDB-like functionality, authentication.

\begin{figure}
\begin{center}
 \includegraphics[width=0.5\textwidth]{images/catopt1}
\caption{Qserv architecture as baselined in \citeds{LDM-135} and currently being implemented \label{fig:catopt1}}.
\end{center}
\end{figure}


\subsection{Heterogeneous Data Access}
 An alternative heterogeneous approach has been proposed which could be followed if we assume that most SQL like queries would be on the Object catalog, or perhaps even the Object-Lite catalog.  Then the Source and other large tables would be stored in something like Parquet files, and accessed with one of the map reduce type systems such as Spark or Dask. This is depicted in \figref{fig:catopt2}.

\begin{figure}
\begin{center}
 \includegraphics[width=0.9\textwidth]{images/catopt2}
\caption{ Heterogeneous catalog access with SQL for the Object catalog but files only for Sources \label{fig:catopt2}}
\end{center}
\end{figure}

We have to be clear SQL would then be restricted to what could fit in a DB (petabyte DBs are possible today.)
There are a few astronomy oriented database implementations which could handle the 40TB Object Lite catalog e.g. SqlServer or Postgres. These also have implementations for MyDB and VO protocols like TAP.

MyDB would then be MyDB .. not in Spark\footnote{\url{https://spark.apache.org/}} or Dask\footnote{\url{https://dask.org/}} - it would be query-able of course.

There may be a higher cost on hardware for this approach, as it will be less efficient for supporting concurrent access to the source table by more than a few users at a time.

\section {Recent Developments}

Since the baseline plan has not to date been altered, Qserv development has continued per plan.  A couple of recent developments are worthy of note, however:

\begin{itemize}

\item The DRP pipelines group has recently adopted Parquet as the file format on record for intermediate data products.

\item Preliminary response from reviewers and early adopters of the LSP has been overwhelmingly positive with respect to the possibility of using the Dask framework for notebook-based analysis of catalog data products.  This, in combination with the above item, make it seem both desirable and likely that DRP products be made available in Parquet format in addition to any other plans.

\item Investigations into implementation strategies for the baselined (but very underspecified) "next-to-db" functionality have yielded the conclusion that users want and expect a level of sophistication in non-SQL / programmatic access to data products that already exists in popular off-the-shelf solutions but would be difficult to achieve in a project-built, tightly integrated solution.

\item A proof-of-concept collaborative exploration with Google was conducted \citedsp{DMTN-078}, in which Qserv was demonstrated to be both practically field-able and reasonably performant on Google cloud infrastructure.  During this exploration, "shoot-outs" were also conducted between Qserv and Google BigQuery. The results were a mixed bag; each technology out-performing the other on certain types of queries in terms of cost and performance \citedsp{Document-31100}. There is apparently yet still some life to be had from an MPP database system.

\end{itemize}

\section {Conclusion}

The above developments taken together would seem to indicate that DM will need to field something architecturally similar to the heterogeneous design presented above, \textit{regardless} of the choice of particular SQL engine on the left (Qserv or non) and/or the extent of data products to be hosted within (Object/Object-Lite vs. the whole ball of wax).  The right-hand side of the heterogeneous design is, in fact, currently being pursued in the context of LSP/LDF development.

With construction nearing completion, and considering the staffing for Qserv development (3.5 FTE), the \textit{management-only} recommendation would be not to change db horses at this time and to cross the start line into operations according to current plans.  If real usage patterns then indicate SQL-oriented access is in fact effectively limited to just the Object table, we should be prepared to swap the database on the left-hand side for something more ideal for operating at that reduced scale; any database hardware resources thus freed could then be absorbed into the computation platform on the right.

A final reason to continue with Qserv is the unique ability to query the Source table - the community have never had such a facility and it may be a route to discovery.  if we agree with \tabref{tab:use} up to 50\% of our science potential lies in that table so providing this as an experiment in DR1 would seem a reasonable investment in potential science return.

If we wished to reduce risk, and in the era of in-kind contributions, having a Postgres or other implementation of ObjectLite available with a TAP interface would be a super contribution - this may bring its own interop problems with MyDB etc. of course.

Should we agree with \tabref{tab:use} the project would benefit greatly from having multiple sources of Object table available from partner institutions. This would allow us to distribute 95\% of our query load to other locations and allow scientists the easiest access to the most frequently consulted data.
An open data policy would have made this easy to achieve, a data rights driven policy should still consider opening the object catalog up - consider it a sort of advertisement at the cost of perhaps allowing non data rights holders a shot at 20\% of the science discoveries.



\appendix
% Include all the relevant bib files.
% https://lsst-texmf.lsst.io/lsstdoc.html#bibliographies
\section{References} \label{sec:bib}
\bibliography{lsst,lsst-dm,refs_ads,refs,books}

%Make sure lsst-texmf/bin/generateAcronyms.py is in your path
\section{Acronyms used in this document}\label{sec:acronyms}
\addtocounter{table}{-1}
\begin{longtable}{|l|p{0.8\textwidth}|}\hline
\textbf{Acronym} & \textbf{Description}  \\\hline

DB & DataBase \\\hline
DM & Data Management \\\hline
LDM & LSST Data Management (Document Handle) \\\hline
LSST & Large Synoptic Survey Telescope \\\hline
PST & Project Science Team \\\hline
PSTN & Project Science Technical Note \\\hline
Qserv & Proprietary LSST Database system \\\hline
SLAC & No longer an acronym; formerly Stanford Linear Accelerator Center \\\hline
SQL & Structured Query Language \\\hline
TAP & Table Access Protocol \\\hline
\end{longtable}

\end{document}
