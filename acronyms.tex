\addtocounter{table}{-1}
\begin{longtable}{|l|p{0.8\textwidth}|}\hline
\textbf{Acronym} & \textbf{Description}  \\\hline

AMCL & AURA Management Council for LSST \\\hline
Baseline & The point at which project designs or requirements are considered to be 'frozen' and after which all changes must be traced and approved \\\hline
DB & DataBase \\\hline
DM & Data Management \\\hline
DRP & Data Release Production \\\hline
FTE & Full Time Equivalent \\\hline
LDF & LSST Data Facility \\\hline
LDM & LSST Data Management (Document Handle) \\\hline
LSE & LSST Systems Engineering (Document Handle) \\\hline
LSP & LSST Science Platform \\\hline
LSST & Large Synoptic Survey Telescope \\\hline
MPP & Massively Parallel Process \\\hline
Object & In LSST nomenclature this refers to an astronomical object, such as a star, galaxy, or other physical entity. E.g., comets, asteroids are also Objects but typically called a Moving Object or a Solar System Object (SSObject). One of the DRP data products is a table of Objects detected by LSST which can be static, or change brightness or position with time. \\\hline
PST & Project Science Team \\\hline
PSTN & Project Science Technical Note \\\hline
Qserv & Proprietary Database built by SLAC for LSST \\\hline
SLAC & No longer an acronym; formerly Stanford Linear Accelerator Center \\\hline
SQL & Structured Query Language \\\hline
Source & A single detection of an astrophysical object in an image, the characteristics for which are stored in the Source Catalog of the DRP database. The association of Sources that are non-moving lead to Objects; the association of moving Sources leads to Solar System Objects. (Note that in non-LSST usage "source" is often used for what LSST calls an Object.) \\\hline
TAP & Table Access Protocol \\\hline
VO & Virtual Observatory \\\hline
\end{longtable}
