\addtocounter{table}{-1}
\begin{longtable}{|l|p{0.8\textwidth}|}\hline
\textbf{Acronym} & \textbf{Description}  \\\hline

AMCL & AURA Management Council for LSST \\\hline
Baseline & The point at which project designs or requirements are considered to be 'frozen' and after which all changes must be traced and approved \\\hline
Center & An entity managed by AURA that is responsible for execution of a federally funded project \\\hline
DB & DataBase \\\hline
DM & Data Management \\\hline
DMTN & DM Technical Note \\\hline
DRP & Data Release Production \\\hline
Data Management & The LSST Subsystem responsible for the Data Management System (DMS), which will capture, store, catalog, and serve the LSST dataset to the scientific community and public. The DM team is responsible for the DMS architecture, applications, middleware, infrastructure, algorithms, and Observatory Network Design. DM is a distributed team working at LSST and partner institutions, with the DM Subsystem Manager located at LSST headquarters in Tucson. \\\hline
Document & Any object (in any application supported by DocuShare or design archives such as PDMWorks or GIT) that supports project management or records milestones and deliverables of the LSST Project \\\hline
FTE & Full Time Equivalent \\\hline
Handle & The unique identifier assigned to a document uploaded to DocuShare \\\hline
LDF & LSST Data Facility \\\hline
LDM & LSST Data Management (Document Handle) \\\hline
LSE & LSST Systems Engineering (Document Handle) \\\hline
LSP & LSST Science Platform \\\hline
LSST & Large Synoptic Survey Telescope \\\hline
MPP & Massively Parallel Process \\\hline
Object & In LSST nomenclature this refers to an astronomical object, such as a star, galaxy, or other physical entity. E.g., comets, asteroids are also Objects but typically called a Moving Object or a Solar System Object (SSObject). One of the DRP data products is a table of Objects detected by LSST which can be static, or change brightness or position with time. \\\hline
PST & Project Science Team \\\hline
PSTN & Project Science Technical Note \\\hline
Project Science Team & an operational unit within LSST that carries out specific scientific performance investigations as prioritized by the Director, the Project Manager, and the Project Scientist. Its membership includes key scientists on the Project who provide specific necessary expertise. The Project Science Team provides required scientific input on critical technical decisions as the project construction proceeds \\\hline
Qserv & Proprietary Database built by SLAC for LSST \\\hline
SLAC & No longer an acronym; formerly Stanford Linear Accelerator Center \\\hline
SQL & Structured Query Language \\\hline
Source & A single detection of an astrophysical object in an image, the characteristics for which are stored in the Source Catalog of the DRP database. The association of Sources that are non-moving lead to Objects; the association of moving Sources leads to Solar System Objects. (Note that in non-LSST usage "source" is often used for what LSST calls an Object.) \\\hline
TAP & Table Access Protocol \\\hline
VO & Virtual Observatory \\\hline
\end{longtable}
